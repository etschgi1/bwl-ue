\documentclass[11pt]{scrartcl}

% Standard packages
\usepackage[utf8]{inputenc}  % Input in UTF-8
\usepackage[T1]{fontenc}  % Output in T1 fonts (westeuropäische Codierung)
\usepackage{lmodern}  % latin modern fonts
\usepackage[ngerman]{babel}  % deutsches Sprachpaket, neue Rechtschreibung

% Seitensetup
\usepackage{scrlayer-scrpage}  % Seitenformatierung durch KOMA-interne Optionen
\usepackage[top=4cm, bottom=4cm]{geometry}  % Seitengeometrie (kann durch KOMA ersetzt werden, hab ich aber nicht geschafft)
\usepackage[hypcap=false]{caption, subcaption}  % caption editing - hypcap warning with hyperref
\usepackage{array}  % table editing

% additional packages
\usepackage{amsmath, amssymb, amstext}  % math packages (American Math Society)
\usepackage{icomma}  % Kommata in Dezimalzahlen verursachen keinen Abstand mehr
\usepackage{graphicx}  % Bilder einfügen
\usepackage{float} %Bilder placement
\usepackage{pdfpages}  % PDF als vollständige Seiten einfügen
\usepackage{lastpage}  % referenziert die letzte Seite
\usepackage[separate-uncertainty=true]{siunitx}  % bessere Darstellung von Einheiten
\usepackage{makecell,booktabs} %Dicke Tabellenstriche
\usepackage{wrapfig} %allow figure wrap
%\usepackage{datatool}
\usepackage[hidelinks]{hyperref}  % hyperref verlinkt Referenzen - hidelinks entfernt borders um links
% allows for temporary adjustment of side margins
\usepackage{chngpage}
\usepackage{footnote} %footnote package

%misc
\newcommand*{\invisiblepar}{{\setlength{\parfillskip}{0pt}\par}\vskip-\parskip\noindent\ignorespaces}
\usepackage{blindtext}
% package setups
% Kopf- und Fußzeile durch KOMA
\pagestyle{scrheadings}  % KOMA darf entscheiden
\clearpairofpagestyles  % reset
\setkomafont{pageheadfoot}{\normalfont}  % Standardschrift in Kopf- und Fußzeile
\captionsetup{format=plain, font=small, labelfont=bf} %Better caption, Abbildung ist FETT
%\setlength{\headheight}{27.2pt}  % benötigte Höhe Kopfzeile (warning von scrlayer-scrpage, wird aber automatisch so gerendert, falls diese Option weggelassen wird)
\ihead{Formel aus BWL UE}  % Kopf links 
\chead{\textsc{Wachmann} Elias\\12004232}  % Kopf Mitte 
\ohead{\today}  % Kopf rechts 
\cfoot{\pagemark \, / \pageref{LastPage}}  % Fuß Mitte
%noindent after figure:
\AfterEndEnvironment{figure}{\noindent\ignorespaces}
%Setup for footnotes in tables
\makesavenoteenv{tabular}
%Overbar setup
\newcommand{\overbar}[1]{\mkern 1.5mu\overline{\mkern-1.5mu#1\mkern-1.5mu}\mkern 1.5mu}
%Roman numerals
\newcommand{\RNum}[1]{\uppercase\expandafter{\romannumeral #1\relax}}
% Table of Contents
\DeclareTOCStyleEntry{dottedtocline}{section}  % KOMA intern - Inhaltsverzeichnis mit Punkten (nur sections)
% SI
\sisetup{locale = DE}  % deutschsprachige SI-Konvention
\DeclareSIUnit\permille{\text{\textperthousand}} %for permille Symbol
% array
\renewcommand{\arraystretch}{1.2}

\begin{document}

% \includepdf{Deckblatt.pdf} %Uni kein Deckblatt
%Title
\title{Formeln aus BWL UE}
\author{\textsc{Wachmann} Elias\\12004232}
\date{\today}
\maketitle

\tableofcontents
\newpage
\section{Unternehmensbeurteilung}
\subsection{Kenngrößen des Quick-Tests:}

\begin{equation}
    \text{Eigenkapitalquote} = \frac{\text{Eigenkapital}}{\text{Gesamtkapital}}
\end{equation}
\begin{equation}
    \text{Schuldentilgungsdauer (in Jahren)} = \frac{\text{Fremdkapital} - \text{Liquide Mittel}}{\text{Praktiker Cashflow}}
\end{equation}
\begin{equation}
    \text{Gesamtkapitalrentabilität} = \frac{\text{Gewinn (vor Steuer)} + \text{Fremdkapitalzinsen}}{\text{Gesamtkapital}}
\end{equation}
\begin{equation}
    \text{Cashflow-Leistungsrate} = \frac{\text{Praktiker Cashflow}}{\text{Betriebsleistung}} 
\end{equation}
Der Praktiker-Cashflow errechnet sich wie folgt: 
\begin{multline*}
    \text{Praktiker-Cashflow} = \text{Jahresüberschuss} +\text{Abschreibungen im AV} - \text{Zuschreibungen im AV}\\ + \text{Erhöhung langf. Rückst.} - \text{Senkung langf. Rückst.} + \text{Buchwertabgang}
\end{multline*}
\subsection{Strukturbilanz}
\textbf{TODO}
\subsection{Fristgerechte Strukturbilanz}
\textbf{TODO}
\subsection{Jahresabschlusanalyse / Finanzwirtschaftliche Analyse}
\subsubsection{Investitions- \& Vermögensstrukturanalyse}
\textbf{Intensitätskennzahlen}
\begin{equation}
    \text{Anlageintensität} = 100\cdot\frac{\text{Anlagevermögen}}{\text{Gesamtvermögen}} \hspace*{1cm} \text{bereinigte Werte verwenden!}
\end{equation}
\begin{equation}
    \text{Umlaufvermögensintensität} = 100\cdot\frac{\text{Umlaufvermögen}}{\text{Gesamtvermögen}} \hspace*{1cm} \text{bereinigte Werte verwenden!}
\end{equation}
\begin{equation}
    \text{Vorratsintensität} = 100\cdot\frac{\text{Vorräte}}{\text{Gesamtvermögen}} 
\end{equation}
\textbf{Umschlagskoeffizienten}
\begin{multline}
    \text{Umschlagshäufigkeit} = \frac{\text{Umsätze inkl. Ust (20 \%)}}{\varnothing\text{gebundenes Vermögen}} \\ \text{mit} \hspace*{1cm} \varnothing\text{gebundenes Vermögen} = \frac{\text{Vermögen im Jahr } x_1+\text{Vermögen im Jahr } x_2}{2}
\end{multline}
\begin{equation}
    \text{Umschlagsdauer} = \frac{365}{\text{Umschlagshäufigkeit}}
\end{equation}
\begin{equation}
    \text{Umschlagshäufigkeit der Debitoren} = \frac{\text{Umsätze inkl. Ust (20 \%)}}{\varnothing\text{Bestand Forderungen aus L\&L}}
\end{equation}
\begin{equation}
    \text{Umschlagdauer der Debitoren} = \frac{365}{\text{Umschlagshäufigkeit der Debitoren}}
\end{equation}
\begin{equation}
    \text{Umschlagshäufigkeit der Kreditoren} = \frac{\text{Umsätze inkl. Ust (20 \%)}}{\varnothing\text{Bestand Verbindlichkeiten aus L\&L }}
\end{equation}
\begin{equation}
    \text{Umschlagdauer der Kreditoren} = \frac{365}{\text{Umschlagshäufigkeit der Kreditoren}}
\end{equation}
\begin{equation}
    \text{Umschlagshäufigkeit der Vorräte} = \frac{\text{Vorratsverbrauch}}{\varnothing\text{Bestand der Vorräte }}
\end{equation}
\begin{equation}
    \text{Umschlagdauer der Vorräte} = \frac{365}{\text{Umschlagshäufigkeit der Vorräte}}
\end{equation}
\begin{equation}
    \text{Sachanlagenabschreibungsquote} = 100\cdot\frac{\text{planmäßige Abschreibung}}{\frac{\text{historische AK}}{\text{HK}} + \text{Zugänge} \pm \text{Umbuchungen}}
\end{equation}
\begin{multline}
    \text{Sachanlagenabnutzungsgrad} = 100\cdot\frac{\text{kummulierte Abschreibung}-\text{Zuschreibungen}}{\frac{\text{historische AK}}{\text{HK}} + \text{Zugänge} \pm \text{Umbuchungen}}
\end{multline}
\begin{multline}
    \text{Sachanlageninvestitionsquote} = 100\cdot\frac{\text{Nettoinvestitions ins Anlagevermögen}}{\text{Anfangsbestand Anlagevermögen zu Buchwerten}}
\end{multline}
\begin{multline}
    \text{Investitionsdeckung der Anlagen} = 100\cdot\frac{\text{Nettoinvestitions ins Anlagevermögen}}{\text{planmäßige Abschreibung}}
\end{multline}
\subsubsection[]{Finanzierungs- \& Kapitalstrukturanalyse}
\textbf{Eigen- \& Fremdkapitalquote}
\begin{equation}
    \text{Eigenkapitalquote} = 100 \cdot\frac{\text{Eigenkapital}}{\text{Gesamtkapital}}
\end{equation}
\begin{equation}
    \text{Fremdkapitalquote} = 100 \cdot\frac{\text{Fremdkapital}}{\text{Gesamtkapital}}
\end{equation}
\textbf{Verschuldungsgrad \& Gearing}
\begin{equation}
    \text{Verschuldungsgrad} = 100 \cdot\frac{\text{Fremdkapital}}{\text{Eigenkapital}}
\end{equation}
oder
\begin{equation}
    \text{Verschuldungsgrad} = \frac{\text{Nettoverschuldung}}{\text{Eigenkapital}}
\end{equation}
wobei man die Nettoverschuldung aus Bilanz \& Anhang abliest. 
\subsubsection[]{Liquiditätsanalyse}
\textbf{Kurz- \& Langfristige statische Liquiditätsanalyse}
\begin{equation}
    \text{Anlagendeckungsgrad I} = 100\cdot\frac{\text{Eigenkapital}}{\text{Anlagevermögen}}
\end{equation}
\begin{equation}
    \text{Anlagendeckungsgrad II} = 100\cdot\frac{\text{Eigenkapital+langfristiges Fremdkapital}}{\text{Anlagevermögen}}
\end{equation}
\begin{equation}
    \text{Anlagendeckungsgrad III} = 100\cdot\frac{\text{Eigenkapital+langfristiges Fremdkapital}}{\text{Langfristige Aktiva}}
\end{equation}
\begin{equation}
    \text{(Net) Working Capital} = \text{kurzfristige Aktiva} - \text{kurzfristige Passiva}
\end{equation}
\begin{equation}
    \text{Liquiditätsgrad I} = 100\cdot\frac{\text{Zahlungsmittel}}{\text{Kurzfristige Passiva}}
\end{equation}
\begin{equation}
    \text{Liquiditätsgrad II} = 100\cdot\frac{\text{Monetäres Umlaufvermögen (= kurzfristige Aktiva - Vorräte)}}{\text{Kurzfristige Passiva}}
\end{equation}
\begin{equation}
    \text{Liquiditätsgrad III} = 100\cdot\frac{\text{kurzfristige Aktiva}}{\text{Kurzfristige Passiva}}
\end{equation}
\subsubsection[]{Dynamische Liquiditätsanalyse}
\begin{equation}
    \text{Dynamischer Verschuldungsgrad} = \frac{\text{Fremdkapital}}{\text{Praktiker}-\text{Cashflow}}
\end{equation}
\begin{equation}
    \text{Schuldentilgungsdauer} = \frac{\text{Fremdkapital - liquide Mittel}}{\text{Praktiker}-\text{Cashflow}}
\end{equation}
\begin{equation}
    \text{Cashflow-Leistungsrate} = \frac{\text{Praktiker-Cashflow}}{\text{Betriebsleistung}}
\end{equation}
\subsection[]{Erfolgswirtschaftliche Analyse}
\subsubsection[]{Erfolgsanalyse}
\textbf{Einnahmenstrukturanalyse}
\begin{equation}
    \textbf{Exportquote} = 100\cdot\frac{\text{Auslandsumsatz}}{\text{Gesamtumsatz}}
\end{equation}
\begin{equation}
    \textbf{Inlandsumsatzquote} = 100\cdot\frac{\text{Inlandsumsatz}}{\text{Gesamtumsatz}}
\end{equation}
\textbf{Aufwandsstrukturanalyse}
\begin{equation}
    \text{Personalaufwandsquote} = 100\cdot\frac{\text{Personalaufwand}}{\text{Betriebsleistung}}
\end{equation}
\begin{equation}
    \text{Personalaufwands-Tangente} = 100\cdot\frac{\text{Personalaufwand}}{\text{Gesamter betrieblicher Aufwand}}
\end{equation}
Gesamter betrieblicher Aufwand ist auf GuV zu entnehmen -> Summe von 5 bis 8
\begin{equation}
    \text{Materialaufwandsquote} = 100\cdot\frac{\text{Materialaufwands}+\text{bezogene Leistungen}}{\text{Betriebsleistung}}
\end{equation}
\begin{equation}
    \text{Materialaufwands-Tangente} = 100\cdot\frac{\text{\text{Materialaufwands} + \text{bezogene Leistungen}}}{\text{Gesamter betrieblicher Aufwand}}
\end{equation}
%done until here
\subsubsection[]{Rentabilitätsanalyse}
\textbf{Klassische Rentabilitätsanalyse}
\begin{equation}
    \text{Umsatzrentabilität} = 100\cdot\frac{\text{Jahresüberschuss vor Steuern}}{\text{Umsatz}}
\end{equation}
\begin{equation}
    \text{Eigenkapitalrentabilität} = 100\cdot\frac{\text{Ergebnis vor Steuern}}{\varnothing\text{Eigenkapital}}
\end{equation}
\begin{equation}
    \text{Gesamtkapitalrentabilität (=ROI)} = 100\cdot\frac{\text{Ergebnis vor Steuern} + \text{Fremdkapitalzinsen}}{\varnothing\text{Gesamtkapital}}
\end{equation}
\textbf{Gewinngrößen}
\begin{enumerate}
    \item EBT = Ergebnis vor Steuern
    \item EBIT = EBT $\pm$ Zinsen \& ähnliche Aufwände
    \item EBITA = EBIT $\pm$ Abschreibungen auf Firmenwerte und Zuschreibungen auf Firmenwerte
\end{enumerate}
\subsection[]{Reorganisationsbedarf im Sinne des URG}
\section{Einkauf / Beschaffung \& Absatz / Marketing}
\section{Finanzierung}
\subsection[]{Unterschiedliche Kreditformen}
\subsection[]{Cashflow Berechnung}
\subsection[]{Factoring}
\subsection[]{Zero-Bonds}
\subsection[]{Aktien}
\subsection[]{Operation Blanche}
\section{Investition}
\subsection[]{Amortisation}
\subsection[]{Dynamische Amortisation}
\end{document}